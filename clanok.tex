\documentclass[10pt,oneside,slovak,a4paper]{article}

\usepackage[slovak]{babel}
%\usepackage[T1]{fontenc}
\usepackage[IL2]{fontenc} % lepšia sadzba písmena Ľ než v T1
\usepackage[utf8]{inputenc}
\usepackage{graphicx}
\usepackage{url} % príkaz \url na formátovanie URL
\usepackage{hyperref} % odkazy v texte budú aktívne (pri niektorých triedach dokumentov spôsobuje posun textu)
\usepackage{multirow}
\usepackage{cite}
%\usepackage{times}

\pagestyle{myheadings}

\title{Využitie rozšírenej reality vo vzdelaní\thanks{Semestrálny projekt v predmete Metódy inžinierskej práce, ak. rok 2020/21, vedenie: Ing. Fedor Lehocki, PhD}} 

\author{Jakub Smorada\\[2pt]
	{\small Slovenská technická univerzita v Bratislave}\\
	{\small Fakulta informatiky a informačných technológií}\\
	{\small \texttt{xsmorada@stuba.sk}}
	}

\date{\small 17. október 2020} 



\begin{document}

\maketitle

\begin{abstract}
Rozšírená realita je technológia, v ktorej vystupujú reálne prvky a počítačovo vygenerované prvky, viažuce sa k nejakej polohe alebo aktivite.
Rozšírené realita ponúka to najlepšie z reálneho a digitálneho sveta. 
Za posledné roky sa rozšírená realita veľmi rozšírila, pretože ju podporuje väčšina smartfónov. 
Rozšírená realita má rôzne využitia. 
Môže byť využitá v medicíne, armáde, turizme, hrách alebo aj vo vzdelaní. 
Jej výhoda je, že v reálnom prostredí dokáže zobraziť niečo, čo sa v danom prostredí skutočne nevyskytuje. 
Tým pádom aplikácia s rozšírenou realitou vytvára takmer fyzický kontakt medzi študentom a prostredím s rozšírenou realitou, pretože študent vidí reálne prostredie obohatené o prostredie vygenerované počítačom. 
To mu umožňuje napríklad pozrieť sa na dané prostredie z rôznych uhlov. 
Vo vzdelaní môže rozšírená realita pomôcť študentom lepšie a ľahšie sa učiť, nakoľko okrem písaných materiálov takto majú k dispozícii aj obraz, a to v realistickej a 3D forme. 
V tomto článku preskúmam, akú výhodnosť a efektívnosť má rozšírená realita vo vzdelaní a v akých oblastiach vzdelania môže byť aplikovateľná.
\end{abstract}

\section{Úvod}
Technológia stále napreduje a inak to nie je ani pri rozšírenej realite.  
O rozšírenej realite sa hovorí už od 90. rokov, no až teraz, vďaka smartfónom, sa stáva skutočnosťou.\cite{Yuen2011} 
Keď sa povie rozšírená realita, myslí sa tým široká škála technológii, ktoré dokážu premietať počítačovo vytvorený obsah do prostredia reálneho sveta.\cite{Yuen2011}
Tieto technológie nám dokážu sprístupniť 2D a 3D objekty, ktoré sú späté s nejakou aktivitou alebo lokalitou. 
Okrem týchto objektov nám dokážu zobraziť aj iné digitálne prvky, ako napríklad audio alebo video súbory.\cite{Yuen2011} 
Rozšírená realita má aplikáciu v širokej škále rôznych sfér. V tomto článku sa budem zaoberať hlavne oblasťou vzdelania. 
Technológie vo vzdelaní môžu ovplyvniť študentov pri aktívnom vzdelávaní sa a môžu ich motivovať, čo vedie k efektívnemu procesu učenia sa.\cite{Saidin2015}
Predošlé štúdie však hovoria, že zaradenie technológii do vzdelávania vytvorí pasívne učenie, ak používané technológie nevytvoria podmienky pre kritické myslenie študentov.\cite{Saidin2015}
Tu nastupuje rozšírená realita.
Od svojho uvedenia preukazuje rozšírená realita dobrý potenciál na vytvorenie podmienok pre aktívne, efektívne a zmysluplné učenie.\cite{Saidin2015}
\\\\V prvom rade sa budem venovať základnému problému, ktorý bol naznačený v úvode\ref{zakladnyProblem}, kde rozoberiem prečo sú technológie vo vzdelaní dôležité.
Ďalej sa pozriem na učenie sa s technológiami (technology-enhanced learning)\ref{tel}.
A následne rozoberiem konkrétnu technológiu, ktorá má vo vzdelaní veľký potenciál - rozšírenú realitu\ref{rozsirenaVzdelanie}.


\section{Základný problém} \label{zakladnyProblem}
Vo výskume uskutočnenom Teohom a Neom (2007) viacerí respondenti označili za nezaživné, keď iba počúvali svojho vyučujúceho, ktorý im prednášal učivo.\cite{Saidin2015}
Študenti verili, že integrácia technológii do vzdelávania by im pomohla vo vzbudení záujmu o učenie a taktiež samotnom učení. 
Učitelia sa preto rozhodli vyhľadať technológie, ktoré sa dajú integrovať do vzdelávania a tým pádom môžu pomôcť študentom v aktívnom učení sa a lepšom porozumení preberaného učiva.
Obzvlášť mladí ľudia milujú moderné technológie, tak prečo ich nevyužiť nielen pre zábavu, ale aj pre účely vzdelávania? 
Študenti nie sú spokojní so zastaranými spôsobmi, ktoré vo vzdelávaní máme už niekoľko desiatok rokov.
Preto je integrácia technológií do vzdelania potrebná, ba až žiaduca. 
\subsection{Študenti a problém predstavovania si abstraktných konceptov} \label{zakladnyProblem:abstraktneKoncepty}
Študenti majú problém s vednými predmetmi (napr. chémia, fyzika), pretože sú príliš abstraktné a na vizualizáciu jednotlivých učív je potrebná dobrá predstavivosť.\cite{Saidin2015}
To vedie k tomu, že študenti si dané učivo vysvetlia zle, pretože ho nedokážu dostatočne dobre pochopiť.
Preto je výber správnej metódy učenia pre učiteľa kľúčový.
Na tento problém sa dajú využiť rôzne vizualizačné technológie, ktoré majú veľký potenciál znížiť nepochopenie u študentov.
Medzi vizualizačné technológie, ktoré boli preskumané a preukazali veľkú úspešnosť, patria animácie, virtuálne prostredia a simulácie.
Kozhevnikov a Thornton zistili, že je možné zlepšiť predstavivosť študentov pomocou prezentovania abstraktných obrázkov, s ktorými môžu študenti manipulovať.
Robertson a jeho kolegovia zistili, že animácie spolu so zaujímavými dátami a dobrým učiteľom pomáhajú študentom udržať pozornosť a záujem.
\subsection{Technológie pre vizualizáciu abstraktných konceptov} \label{zakladnyProblem:technologie}
Vizualizačné technológie umožňujú študentom vidieť veci, reakcie, deje, ktoré sú príliš rýchle, pomalé, malé alebo veľké.\cite{Saidin2015}
Napríklad v chémii to študentom umožňuje vidieť jednotlivé reakcie medzi molekulami.
Tieto procesy sú abstraktné a bez vizualizačných technológii ťažké na predstavenie.
\\\\
Vizualizačné technológie teda predstavujú nový spôsob, akým sa študenti môžu vzdelávať.
Učenie sa s vizualizačnými technológiami je pre študentov oveľa záživnejšie ako len čítanie si teórie, a zároveň im takáto výučba zaručí udržanie pozornosti na dlhšiu dobu.
Medzi vizualizačné technológie patria napríklad videá, animácie, virtuálna realita alebo rozšírená realita.

\section{Technology-enhanced learning (TEL)} \label{tel}
Technology Enhanced Learning (skratka TEL) predstavuje využitie technológií ako napríklad počítačov, smartfónov, interaktívnej tabule, rozšírenej reality a pod. v rámci procesu vzdelávania, na zefektívnenie a modernizáciu tohto procesu.\cite{TEL} 
TEL môže zlepšiť pochopenie u študentov, ponúknuť im väčšiu flexibilitu alebo taktiež podporiť spoluprácu a komunikáciu. 
Okrem toho TEL podporuje aj rozvoj počítačových zručností, ktoré táto digitálna doba vyžaduje. 
Môže byť použitý v rámci triedy, ako kombinované vzdelávanie alebo aj na dištančné štúdium.
\subsection{Predpoklady pre TEL} \label{tel:predpoklady}
TEL v sebe ukrýva veľký potenciál.\cite{Daniela2018}  
Jeho využitie je však podmienené istými faktormi, a to najmä kompetenciami učiteľa s danými technológiami narábať, a efektívne ich používať tam, kde je to vhodné a slúžiace zlepšeniu študijných výsledkov.
Taktiež môžeme spomenúť isté kompetencie vyžadujúce sa od študentov, no v tomto platí pravidlo, čím mladšia generácia, tým zdatnejšia v oblasti IKT. 
A v neposlednom rade je nevyhnutná spolupráca samotného vedenia školy, ktoré zaobstará potrebné vybavenie do školských zariadení. 
Okrem tejto technickej podpory by malo uskutočniť aj podporu administratívnu, t. j. rôzne školenia zamerané na rozvoj kritického myslenia a správne využitie digitálnych technológií ako pre zábavu, tak aj pre štúdium.
\\\\V dnešnej dobe sa rozšírená realita ponúka ako najlepšia vizualizačná technológia pre vzdelanie.
Rozšírená realita je nová technológia, ktorá bude mať v budúcnosti s najväčšou pravdepodobnosťou významný dopad na vzdelanie.\cite{Saidin2015} 
\section{Rozšírená realita vo vzdelaní} \label{rozsirenaVzdelanie}
Ako som už vyššie spomínal, rozšírená realita má naozaj mnoho využití. 
Poďme sa pozrieť na jej využitie vo vzdelaní.\\\\
Z mojej skúsenosti študenti radi skúšajú nové metódy výučby.
A teda nie je prekvapením, že rôzne štúdie dokázali, že študenti si pochvaľujú túto novú technológiu vo vzdelaní.\cite{Saidin2015}
Vďaka rozšírenej realite a jej interaktívným aplikáciam sa študenti učia aktívnejšie.
To teda prispieva aj ku kritickému a kreatívnemu rozmýšľaniu študentov.

\begin{table}[h]
	\begin{tabular}{|l|l|}
	\hline
	Autor                 & \multicolumn{1}{c|}{Výhody rozšírenej reality}                                                                                                         \\ \hline
	Singhal et al. (2012) & \begin{tabular}[c]{@{}l@{}}Podporuje interakciu medzi skutočným a virtuálnym\\ prostredím a tým umožňuje manipuláciu objektov v priestore\end{tabular} \\ \hline
	Coffin et al. (2008)  & Ponúka učiteľom spôsob ako zlepšiť chápanie u študentov                                                                                                \\ \hline
	Burton et al. (2011)  & Ponúka študentom lepšie možnosti učenia aj mimo školy                                                                                                  \\ \hline
	\end{tabular}
\caption{Výhody rozšírenej reality\cite{Saidin2015}}
\label{rozsirenaVyhody}
\end{table}

V tabuľke \ref{rozsirenaVyhody} môžeme vidieť, aké výhody rozšírenej reality vnímajú rôzni autori zaoberajúci sa touto tématikou.
Integrovanie rozšírenej reality môže vzdelávaniu vdychnúť nový život.
\subsection{Discovery-based learning (učenie sa založené na objavovaní)} \label{rozsirenaVzdelanie:dbl}

\subsection{Modelovanie objektov} \label{rozsirenaVzdelanie:modelovanie}

\subsection{Trénovanie zručností} \label{rozsirenaVzdelanie:zrucnosti}

\section{Budúcnosť rozšírenej reality a vzdelania} \label{rozsirenaBuducnost}


\section{Záver} \label{zaver} % prípadne iný variant názvu

% týmto sa generuje zoznam literatúry z obsahu súboru literatura.bib podľa toho, na čo sa v článku odkazujete
\bibliography{literatura}
\bibliographystyle{plain} % prípadne alpha, abbrv alebo hociktorý iný
\end{document}
