% Metódy inžinierskej práce

\documentclass[10pt,twoside,slovak,a4paper]{article}

\usepackage[slovak]{babel}
%\usepackage[T1]{fontenc}
\usepackage[IL2]{fontenc} % lepšia sadzba písmena Ľ než v T1
\usepackage[utf8]{inputenc}
\usepackage{graphicx}
\usepackage{url} % príkaz \url na formátovanie URL
\usepackage{hyperref} % odkazy v texte budú aktívne (pri niektorých triedach dokumentov spôsobuje posun textu)

\usepackage{cite}
%\usepackage{times}

\pagestyle{headings}

\title{Využitie rozšírenej reality vo vzdelaní\thanks{Semestrálny projekt v predmete Metódy inžinierskej práce, ak. rok 2020/21, vedenie: Ing. Fedor Lehocki, PhD}} 

\author{Jakub Smorada\\[2pt]
	{\small Slovenská technická univerzita v Bratislave}\\
	{\small Fakulta informatiky a informačných technológií}\\
	{\small \texttt{xsmorada@stuba.sk}}
	}

\date{\small 17. október 2020} 



\begin{document}

\maketitle

\begin{abstract}
Rozšírená realita je technológia, v ktorej vystupujú reálne prvky a počítačovo vygenerované prvky, viažuce sa k nejakej polohe alebo aktivite.\cite{Yuen2011} 
Rozšírené realita ponúka to najlepšie z reálneho a digitálneho sveta. 
Za posledné roky sa rozšírená realita veľmi rozšírila, pretože ju podporuje väčšina smartfónov. 
Rozšírená realita má rôzne využitia. 
Môže byť využitá v medicíne, armáde, turizme, hrách alebo aj vo vzdelaní. 
Jej výhoda je, že v reálnom prostredí dokáže zobraziť niečo, čo sa v danom prostredí skutočne nevyskytuje. 
Tým pádom aplikácia s rozšírenou realitou vytvára takmer fyzický kontakt medzi študentom a prostredím s rozšírenou realitou, pretože študent vidí reálne prostredie obohatené o prostredie vygenerované počítačom. 
To mu umožňuje napríklad pozrieť sa na dané prostredie z rôznych uhlov. 
Vo vzdelaní môže rozšírená realita pomôcť študentom lepšie a ľahšie sa učiť, nakoľko okrem písaných materiálov takto majú k dispozícii aj obraz, a to v realistickej a 3D forme. 
V tomto článku preskúmam, akú výhodnosť a efektívnosť má rozšírená realita vo vzdelaní a v akých oblastiach vzdelania môže byť aplikovateľná.
\end{abstract}

\section{Úvod}
Technológia stále napreduje a inak to nie je ani pri rozšírenej realite.  
O rozšírenej realite sa hovorí už od 90. rokov, no až teraz, vďaka smartfónom, sa stáva skutočnosťou.\cite{Yuen2011} 
Keď sa povie rozšírená realita, myslí sa tým široká škála technológii, ktoré dokážu premietať počítačovo vytvorený obsah do prostredia reálneho sveta.\cite{Yuen2011}
Rozšírená realita nám dokáže sprístupniť 2D a 3D objekty, ktoré sú späté s nejakou aktivitou alebo lokalitou. 
Okrem týchto objektov nám dokáže zobraziť aj iné digitálne prvky, ako napríklad audio alebo video súbory.\cite{Yuen2011} 
Rozšírená realita má aplikáciu v širokej škále rôznych sfér. V tomto článku sa budem zaoberať hlavne oblasťou vzdelania. 
Technológie vo vzdelaní môžu ovplyvniť študentov pri aktívnom vzdelávaní sa a môžu ich motivovať, čo vedie k efektívnemu procesu učenia.\cite{Saidin2015}
V prvom rade sa budem venovať pozadiu problému\ref{rozsirenaRealita}.
Motivujte čitateľa a vysvetlite, o čom píšete. Úvod sa väčšinou nedelí na časti.

Uveďte explicitne štruktúru článku. Tu je nejaký príklad.
Základný problém, ktorý bol naznačený v úvode, je podrobnejšie vysvetlený v časti~\ref{nejaka}.
Dôležité súvislosti sú uvedené v častiach~\ref{dolezita} a~\ref{dolezitejsia}.
Záverečné poznámky prináša časť~\ref{zaver}.



\section{Pozadie problému} \label{rozsirenaRealita}
Vo výskume uskutočnenom Teohom a Neom (2007) respondenti hlásili, že je nudné len počúvať svojho vyučujúceho, ktorý prezentuje pred nimi.
Študenti verili, že integrácia technológii do vzdelávania by im pomohla v učení. 
Učitelia sa preto rozhodli vyhľadať technológie, ktoré sa dajú integrovať do vzdelávania a tým pádom môžu pomôcť študentom v aktívnom učení sa a lepšom porozumení preberaného učiva.\cite{Saidin2015}
\subsection{Študenti a problém predstavovania si abstraktných konceptov} \label{rozsirenaRealita:abstraktneKoncepty}

Z obr.~\ref{f:rozhod} je všetko jasné. 

\begin{figure*}[tbh]
\centering
%\includegraphics[scale=1.0]{diagram.pdf}
Aj text môže byť prezentovaný ako obrázok. Stane sa z neho označný plávajúci objekt. Po vytvorení diagramu zrušte znak \texttt{\%} pred príkazom \verb|\includegraphics| označte tento riadok ako komentár (tiež pomocou znaku \texttt{\%}).
\caption{Rozhodujúci argument.}
\label{f:rozhod}
\end{figure*}



\section{Iná časť} \label{ina}

Základným problémom je teda\ldots{} Najprv sa pozrieme na nejaké vysvetlenie (časť~\ref{ina:nejake}), a potom na ešte nejaké (časť~\ref{ina:nejake}).\footnote{Niekedy môžete potrebovať aj poznámku pod čiarou.}

Môže sa zdať, že problém vlastne nejestvuje\cite{Coplien:MPD}, ale bolo dokázané, že to tak nie je~\cite{Czarnecki:Staged, Czarnecki:Progress}. Napriek tomu, aj dnes na webe narazíme na všelijaké pochybné názory\cite{PLP-Framework}. Dôležité veci možno \emph{zdôrazniť kurzívou}.


\subsection{Nejaké vysvetlenie} \label{ina:nejake}

Niekedy treba uviesť zoznam:

\begin{itemize}
\item jedna vec
\item druhá vec
	\begin{itemize}
	\item x
	\item y
	\end{itemize}
\end{itemize}

Ten istý zoznam, len číslovaný:

\begin{enumerate}
\item jedna vec
\item druhá vec
	\begin{enumerate}
	\item x
	\item y
	\end{enumerate}
\end{enumerate}


\subsection{Ešte nejaké vysvetlenie} \label{ina:este}

\paragraph{Veľmi dôležitá poznámka.}
Niekedy je potrebné nadpisom označiť odsek. Text pokračuje hneď za nadpisom.



\section{Dôležitá časť} \label{dolezita}




\section{Ešte dôležitejšia časť} \label{dolezitejsia}




\section{Záver} \label{zaver} % prípadne iný variant názvu



%\acknowledgement{Ak niekomu chcete poďakovať\ldots}


% týmto sa generuje zoznam literatúry z obsahu súboru literatura.bib podľa toho, na čo sa v článku odkazujete
\bibliography{literatura}
\bibliographystyle{plain} % prípadne alpha, abbrv alebo hociktorý iný
\end{document}
