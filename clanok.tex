\documentclass[10pt,twoside,slovak,a4paper]{article}

\usepackage[slovak]{babel}
%\usepackage[T1]{fontenc}
\usepackage[IL2]{fontenc} % lepšia sadzba písmena Ľ než v T1
\usepackage[utf8]{inputenc}
\usepackage{graphicx}
\usepackage{url} % príkaz \url na formátovanie URL
\usepackage{hyperref} % odkazy v texte budú aktívne (pri niektorých triedach dokumentov spôsobuje posun textu)

\usepackage{cite}
%\usepackage{times}

\pagestyle{headings}

\title{Využitie rozšírenej reality vo vzdelaní\thanks{Semestrálny projekt v predmete Metódy inžinierskej práce, ak. rok 2020/21, vedenie: Ing. Fedor Lehocki, PhD}} 

\author{Jakub Smorada\\[2pt]
	{\small Slovenská technická univerzita v Bratislave}\\
	{\small Fakulta informatiky a informačných technológií}\\
	{\small \texttt{xsmorada@stuba.sk}}
	}

\date{\small 17. október 2020} 



\begin{document}

\maketitle

\begin{abstract}
Rozšírená realita je technológia, v ktorej vystupujú reálne prvky a počítačovo vygenerované prvky, viažuce sa k nejakej polohe alebo aktivite.\cite{Yuen2011} 
Rozšírené realita ponúka to najlepšie z reálneho a digitálneho sveta. 
Za posledné roky sa rozšírená realita veľmi rozšírila, pretože ju podporuje väčšina smartfónov. 
Rozšírená realita má rôzne využitia. 
Môže byť využitá v medicíne, armáde, turizme, hrách alebo aj vo vzdelaní. 
Jej výhoda je, že v reálnom prostredí dokáže zobraziť niečo, čo sa v danom prostredí skutočne nevyskytuje. 
Tým pádom aplikácia s rozšírenou realitou vytvára takmer fyzický kontakt medzi študentom a prostredím s rozšírenou realitou, pretože študent vidí reálne prostredie obohatené o prostredie vygenerované počítačom. 
To mu umožňuje napríklad pozrieť sa na dané prostredie z rôznych uhlov. 
Vo vzdelaní môže rozšírená realita pomôcť študentom lepšie a ľahšie sa učiť, nakoľko okrem písaných materiálov takto majú k dispozícii aj obraz, a to v realistickej a 3D forme. 
V tomto článku preskúmam, akú výhodnosť a efektívnosť má rozšírená realita vo vzdelaní a v akých oblastiach vzdelania môže byť aplikovateľná.
\end{abstract}

\section{Úvod}
Technológia stále napreduje a inak to nie je ani pri rozšírenej realite.  
O rozšírenej realite sa hovorí už od 90. rokov, no až teraz, vďaka smartfónom, sa stáva skutočnosťou.\cite{Yuen2011} 
Keď sa povie rozšírená realita, myslí sa tým široká škála technológii, ktoré dokážu premietať počítačovo vytvorený obsah do prostredia reálneho sveta.\cite{Yuen2011}
Tieto technológie nám dokážu sprístupniť 2D a 3D objekty, ktoré sú späté s nejakou aktivitou alebo lokalitou. 
Okrem týchto objektov nám dokážu zobraziť aj iné digitálne prvky, ako napríklad audio alebo video súbory.\cite{Yuen2011} 
Rozšírená realita má aplikáciu v širokej škále rôznych sfér. V tomto článku sa budem zaoberať hlavne oblasťou vzdelania. 
Technológie vo vzdelaní môžu ovplyvniť študentov pri aktívnom vzdelávaní sa a môžu ich motivovať, čo vedie k efektívnemu procesu učenia sa.\cite{Saidin2015}
Predošlé štúdie hovoria, že zaradenie technológii do vzdelávania vytvorí pasívne učenie, ak používané technológie nevytvoria podmienky pre kritické myslenie študentov.\cite{Saidin2015}
No od svojho uvedenia preukazuje rozšírená realita dobrý potenciál na vytvorenie podmienok pre aktívne, efektívne a zmysluplné učenie.\cite{Saidin2015}
V prvom rade sa budem venovať základnému problému, ktorý bol naznačený v úvode\ref{rozsirenaRealita}, a následne sa vyjadrím k problému študentov predstavovať si abstraktné koncepty\ref{rozsirenaRealita:abstraktneKoncepty}.


Motivujte čitateľa a vysvetlite, o čom píšete. Úvod sa väčšinou nedelí na časti.

Uveďte explicitne štruktúru článku. Tu je nejaký príklad.
Základný problém, ktorý bol naznačený v úvode, je podrobnejšie vysvetlený v časti~\ref{nejaka}.
Dôležité súvislosti sú uvedené v častiach~\ref{dolezita} a~\ref{dolezitejsia}.
Záverečné poznámky prináša časť~\ref{zaver}.



\section{Základný problém} \label{rozsirenaRealita}
Vo výskume uskutočnenom Teohom a Neom (2007) viacerí respondenti označili za nezaživné, keď iba počúvali svojho vyučujúceho, ktorý im prednášal učivo.
Študenti verili, že integrácia technológii do vzdelávania by im pomohla vo vzbudení záujmu o učenie a taktiež samotnom učení. 
Učitelia sa preto rozhodli vyhľadať technológie, ktoré sa dajú integrovať do vzdelávania a tým pádom môžu pomôcť študentom v aktívnom učení sa a lepšom porozumení preberaného učiva.\cite{Saidin2015}
\subsection{Študenti a problém predstavovania si abstraktných konceptov} \label{rozsirenaRealita:abstraktneKoncepty}
Študenti majú problém s vednými predmetmi (napr. chémia, fyzika), pretože sú príliš abstraktné a na ich vizualizáciu je potrebná predstavivosť (Gilbert, 2004).
To vedie k tomu, že študenti si dané učivo vysvetlia zle, pretože mu nedokážu dostatočne dobre pochopiť.
Preto je výber správnej metódy učenia pre učiteľa kľúčova (Palmer, 2001).
Na tento problém sa dajú využiť rôzne vizualizačné technológie, ktoré majú veľký potenciál znížiť nepochopenie u študentov (Hay et al., 2000).
Kozhevnikov a Thornton (2007) zistili, že je možné zlepšiť predstavivosť študentov pomocou prezentovania abstraktných obrázkov, s ktorými môžu študenti manipulovať.
Vizualizačné technológie, ktoré boli preskúmané zahrňajú animácie, virtuálne prostredia a simulácie.

\section{Rozšírená realita vo vzdelaní} \label{rozsirenaVzdelanie}

\subsection{Discovery-based learning (učenie sa založené na objavovaní)} \label{rozsirenaVzdelanie:dbl}

\subsection{Modelovanie objektov} \label{rozsirenaVzdelanie:modelovanie}

\subsection{Trénovanie zručností} \label{rozsirenaVzdelanie:zrucnosti}

\paragraph{Veľmi dôležitá poznámka.}
Niekedy je potrebné nadpisom označiť odsek. Text pokračuje hneď za nadpisom.



\section{Budúcnosť rozšírenej reality a vzdelania} \label{rozsirenaBuducnost}


\section{Záver} \label{zaver} % prípadne iný variant názvu



%\acknowledgement{Ak niekomu chcete poďakovať\ldots}


% týmto sa generuje zoznam literatúry z obsahu súboru literatura.bib podľa toho, na čo sa v článku odkazujete
\bibliography{literatura}
\bibliographystyle{plain} % prípadne alpha, abbrv alebo hociktorý iný
\end{document}
